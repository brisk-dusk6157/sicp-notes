% Created 2023-05-17 Wed 15:32
% Intended LaTeX compiler: pdflatex
\documentclass[11pt]{article}
\usepackage[utf8]{inputenc}
\usepackage[T1]{fontenc}
\usepackage{graphicx}
\usepackage{longtable}
\usepackage{wrapfig}
\usepackage{rotating}
\usepackage[normalem]{ulem}
\usepackage{amsmath}
\usepackage{amssymb}
\usepackage{capt-of}
\usepackage{hyperref}
\author{LMG}
\date{\today}
\title{}
\hypersetup{
 pdfauthor={LMG},
 pdftitle={},
 pdfkeywords={},
 pdfsubject={},
 pdfcreator={Emacs 28.2 (Org mode 9.5.5)}, 
 pdflang={English}}
\begin{document}

\tableofcontents

\section{Exercise context}
\label{sec:orgd575b89}
In the Section 1.2.5 SICP discusses Greates Common Divisor and Euclid's
algorithm.

Exercise is to work through the normal-order and applicative-order evaluation.

Here the code:
\begin{verbatim}
#lang sicp

(define (gcd a b)
  (if (= b 0)
      a
      (gcd b (remainder a b))))
(gcd 24 18)
\end{verbatim}

\begin{verbatim}
6
\end{verbatim}


\subsection{Normal-order evaluation:}
\label{sec:org78db0d2}

\begin{verbatim}
 1  (gcd 206
 2  	 40)
 3  (if (= 40 0)
 4      206
 5      (gcd 40
 6  	 (remainder 206 40)))
 7  (gcd 40
 8       (remainder 206 40))
 9  (if (= (remainder 206 40) 0)
10      40
11      (gcd (remainder 206 40)
12  	 (remainder 40 (remainder 206 40))))
13  ;; +1
14  (if (= 6 0)
15      40
16      (gcd (remainder 206 40)
17  	 (remainder 40 (remainder 206 40))))
18  
19  (gcd (remainder 206 40)
20       (remainder 40 (remainder 206 40)))
21  
22  (if (= (remainder 40 (remainder 206 40)) 0)
23      (remainder 206 40)
24      (gcd (remainder 40 (remainder 206 40))
25  	 (remainder (remainder 206 40)
26  		    (remainder 40 (remainder 206 40)))))
27  
28  ;; +2
29  (if (= 4 0)
30      (remainder 206 40)
31      (gcd (remainder 40 (remainder 206 40))
32  	 (remainder (remainder 206 40)
33  		    (remainder 40 (remainder 206 40)))))
34  
35  (gcd (remainder 40 (remainder 206 40))
36       (remainder (remainder 206 40)
37  		(remainder 40 (remainder 206 40))))
38  (if (= (remainder (remainder 206 40)
39  		  (remainder 40 (remainder 206 40)))
40         0)
41      (remainder 206 40)
42      (gcd (remainder (remainder 206 40)
43  		    (remainder 40 (remainder 206 40)))
44  	 (remainder (remainder 40 (remainder 206 40))
45  		    (remainder (remainder 206 40)
46  			       (remainder 40 (remainder 206 40))))))
47  ;; +4
48  (if (= 2 0)
49      (remainder 206 40)
50      (gcd (remainder (remainder 206 40)
51  		    (remainder 40 (remainder 206 40)))
52  	 (remainder (remainder 40 (remainder 206 40))
53  		    (remainder (remainder 206 40)
54  			       (remainder 40 (remainder 206 40))))))
55  
56  (gcd (remainder (remainder 206 40)
57  		(remainder 40 (remainder 206 40)))
58       (remainder (remainder 40 (remainder 206 40))
59  		(remainder (remainder 206 40)
60  			   (remainder 40 (remainder 206 40)))))
61  (if (= (remainder (remainder 40 (remainder 206 40))
62  		  (remainder (remainder 206 40)
63  			     (remainder 40 (remainder 206 40))))
64         0)
65      (remainder (remainder 206 40)
66  	       (remainder 40 (remainder 206 40)))
67      )
68  ;; +7
69  (if (= 0 0)
70      (remainder (remainder 206 40)
71  	       (remainder 40 (remainder 206 40)))
72      (...))
73  (remainder (remainder 206 40)
74  	   (remainder 40 (remainder 206 40)))
75  ;; +2
76  (remainder 6
77  	   (remainder 40 6))
78  ;; +1
79  (remainder 6 4)
80  ;; +1
81  ;; => 2
82  
\end{verbatim}

\begin{verbatim}
#lang sicp
(remainder 206
	   40)
(remainder 40
	   (remainder 206 40))
(remainder (remainder 206 40)
	   (remainder 40 (remainder 206 40)))
(remainder (remainder 40
		      (remainder 206 40))
	   (remainder (remainder 206 40) (remainder 40 (remainder 206 40))))
\end{verbatim}

\begin{verbatim}
6
4
2
0
\end{verbatim}


So the number of calls to \texttt{remainder} is 18:
\begin{verbatim}
(+ 1 2 4 7 2 1 1)
\end{verbatim}

\begin{verbatim}
18
\end{verbatim}

\subsubsection{Normal-order evaluation, \texttt{if}}
\label{sec:org1d8a258}

I'm not sure how the special form \texttt{if} is considered in normal-order evaluation.

p.16 says "would not evaluate the operands until their values were needed". Arguably, \texttt{if} is one such case.

p.21,Ex1.5 points that one can assume that the evaluation rule for the special form \texttt{if} does not depend on order of evaluation: the predicate exp is evaluated first and the result determines whether to evaluate the conseuqent or the alternative expression.

\subsection{?}
\label{sec:orgcb0b3ae}

\subsection{Applicative-order evaluation:}
\label{sec:orga13469e}

\begin{verbatim}
(gcd 206 40)
(gcd 40 (remainder 206 40))
(gcd 40 6)
(gcd 6 (remainder 40 6))
(gcd 6 4)
(gcd 4 (remainder 6 4))
(gcd 4 2)
(gcd 2 (remainder 4 2))
(gcd 2 0)
=> 2
\end{verbatim}
\end{document}